\documentclass[11pt]{article}

\usepackage{xcolor}
\definecolor{light-gray}{gray}{0.90}
\usepackage{listings}
\usepackage{listings}
\lstdefinestyle{Bash}
{	language=bash,
	%keywordstyle=\color{red},
	backgroundcolor=\color{light-gray},
	basicstyle=\ttfamily,
	xleftmargin=.25in,
	xrightmargin=.25in,
	breaklines=true,
	literate={\$}{{\textcolor{blue}{\$}}}1 
}

\begin{document}

\title{JPL Spice on my Mac}
\author{Justin Erwin}
\date{updated \today}
\maketitle

I put all the spice stuff in a directory in my home folder called "naif", so its full path is the "$\sim$/lib/naif". The C and Fortran toolkit can be downloaded from JPL's Spice Toolkit webpage (http://naif.jpl.nasa.gov/naif/toolkit.html).



%%%%%%%
\section{Fortran}

\subsection{Basic install}
\begin{enumerate}
	\item Download the Fortran toolkit for gfortran 64bit, need files "toolkit.tar.Z" and "importSpice.csh", and put them in the "$\sim$/lib/naif" directory.
	\item From the terminal, cd into the "$\sim$/naif" directory, and run: 
\begin{lstlisting}[style=Bash]
/bin/csh importSpice.csh
\end{lstlisting}
	This will extract all the files into a new folder "toolkit".
\end{enumerate}

\subsection{Recompile with local gfortran}
Just to make sure the toolkit is compatible with the code I write, I recompile the code.
\begin{enumerate}
	\item cd into "toolkit"
	\item enter:
\begin{lstlisting}[style=Bash]
/bin/csh makeall.csh
\end{lstlisting} 
	Let it do its work.
\end{enumerate}

\subsection{Linking to Library}
describe how to correctly link to library using gfortran.



%%%%%%%
\section{C}

\subsection{Basic install}
\begin{enumerate}
	\item Download the C toolkit for "Mac/Intel, OSX, Apple C, 64bit", need files "cspice.tar.Z" and "importCSpice.csh", and put them in the "$\sim$/lib/naif" directory.
	\item From the terminal, cd into the "$\sim$/lib/naif" directory, and run 
\begin{lstlisting}[style=Bash]
/bin/csh importCSpice.csh
\end{lstlisting}	
	This will extract all the files into a new folder "cspice".
\end{enumerate}

\subsection{Recompile with local gcc}
Just to make sure the toolkit is compatible with the code I write, I recompile the code.
\begin{enumerate}
	\item cd into "cspice"
	\item Instead of using Apples C compiler (cc), we want our gcc version. To trigger this in the provided script we need to set the environmental variable "TKCOMPILER":
\begin{lstlisting}[style=Bash]
export TKCOMPILER="gcc"
\end{lstlisting} 
	\item enter:
\begin{lstlisting}[style=Bash]
/bin/csh makeall.csh
\end{lstlisting} 
	 Let it do its work.
\end{enumerate}

\subsection{Linking to Library}
describe how to correctly link to library using gcc.



%%%%%%%
\section{Python}

The most complete Python implementation/wrapper of Spice is SpiceyPy, available via github. I used a combination of Macports and Pip to install all the requirements, obtained SpiceyPy using git, and did a manual install using the --user option. In this installation it does download its own copy of CSpice, but who cares about hard-drive space at this point.


\end{document}
